\documentclass{article}

\usepackage{graphicx}
\usepackage{cleveref}
\usepackage{authblk}
\usepackage{amssymb}
\usepackage{float}

\begin{document}

\title{Project in Advanced Multiprocessor Programming}
\author[1]{Aleksei Karasev - 12227646\\}
\author[2]{Mariana da Silva Barros - 12202389}
\affil[1]{MSc student of Computer Engineering, TU Wien}
\affil[2]{MSc student of Computer Engineering, TU Wien}
\maketitle

\section{Introduction}\mbox{}\par
\textit{Priority queues} serve as an important basic tool in algorithmic design.
They are widely used in a variety of applications and systems, such as simulation systems, job scheduling (in computer systems), networking (e.g., routing and realtime bandwidth management), file compression, numerical computations, and
more. With the proliferation of modern parallel platforms, the need for a highperformance concurrent implementation of the priority queue has become acute.\par
A priority queue (PQ) supports two operations: \textit{insert} and \textit{deleteMin}. The
abstract definition of a PQ provides a set of key-value pairs, where the key
represents a priority. The \textit{insert()} method inserts a new key-value pair into the
set (the keys don’t have to be unique), and the \textit{deleteMin()} method removes and
returns the value of the key-value pair with the lowest key (i.e., highest priority)
in the set.\par
In this report we present the design of a high performance lock-free linearizable PQ called CBPQ based on the paper [1].\par
In a nutshell, CBPQ combines the following ideas. We use chunked linked-list as underlying data structure which allows to make search of chunk for insertion operation more efficiently. For threads concurrency we apply syncronization primitives such as critical sections and atomic operation (\textit{atomic or, atomic xor, atomic fetch and increment, atomic compare and swap}).\par
The more detailed description of how CBPQ works will be considered in next section.

\section{Overview}
\subsection{CBPQ Design}\mbox{}\par
As was already mentioned before CBPQ based on chunked linked-list. In particular, queue contains three chunks in the begining.\par
First chunk is the special one which stores key-value pairs with highest priority (key range $0 \leqslant 20$). This is the only chunk which supports \textit{deleteMin} operation and doesn't permit direct insertion. In order to put elements into first chunk we should use assosiated buffer as discussed below.\par
The remaining two chunks serve for insertion only and filled without buffer. Key ranges of second and third chunks are $21 \leqslant 40$ and $41 \leqslant 4294967295$ respectively.\par
Due to lack of space we omit concrete description of chunk structure and its properties which might be found in paper our work based on. [1]\par
Now, we would like to give a brief explanation of \textit{insert} and \textit{deleteMin} operations usage and take a look on several cases which might appear during queue exploitation.

\paragraph{Insertion into the first chunk}\mbox{}\par
Insertion into the first chunk which has state "DELETION" appers once we are about to store key-value pair with key range $0 \leqslant 20$.\par
Right after first chunk was found, \textit{insertToBuffer()} function is called in order to put element into the buffer.
Inside this function each thread checks either buffer was already allocated by one of previous threads or not. If it was then element is stored in already existing buffer, otherwise thread first allocates buffer and stores element into it.\par
Every time new element placed into the buffer we need to restructure CBPQ in order to merge current first chunk with its buffer. First of all chunks are marked as "FROZEN" in \textit{freezeChunk()} function. Doing this we signalize to all other threads that this chunk is already in use, so threads don't mess together trying to use one chunk. Once chunk is frozen, \textit{freezeRecovery()} function is called in order to restructure CBPQ. In case of insertion into the first chunk, we execute \textit{mergeFirstChunk()} function which merges current first chunk with buffer and returns pointer to the new first chunk which is now actual first chunk in our queue. In situation when current first chunk is already full and don't have space for new elements from buffer several "DELETION" chunks are created. Also there might be a case when this is impossible for thread to put new element into buffer due to buffer is completely packed by other threads already. Then thread restructures queue first and then retries to store its element again.\par
Once new key-value pair is successfully added into first chunk thread increases value of index using atomic F\&I instruction and we can safely return.
\pagebreak

\paragraph{Insertion into non-first chunk}\mbox{}\par
Now let's consider insertion into any "INSERTION" chunk. This case appear when key-value pair with key greater than 20 should be stored into queue.\par
Generaly, there are two situation which might appear during insertion into non-first chunk.\par
First case happens when we are going to store element into not fully packed chunk. If it's so, thread just gets actual index of first free position and store element directly into the chunk. After that insirtion operation is finished.\par
The second case is a bit more complicated which appears when current non-first chunk is full and CBPQ should be restructured calling \textit{freezeChunk()} and \textit{freezeRecovery()} functions. These two function's behavior pretty the same as for insertion into the first chunk but couple of moments. Function \textit{freezeChunk()} calls \textit{freezeKeys()} function for every non-first chunk in order to freeze all elements which are stored into the chunk already. Function \textit{freezeRecovery()} invokes \textit{split()} function which splits current full non-first chunk into two new half-full chunks containing lower-valued and higher-valued parts of keys respectively. After CPBQ was restructured thread retries to store element into non-first chunk again.\par
As in case of insertion into the first chunk, once new key-value pair is successfully added thread increases value of index using atomic F\&I instruction and we can safely return.

\paragraph{Deletion from the first chunk}\mbox{}\par
This operation is supported only by the first chunk or any other chunk with state "DELETION".\par
Deletion is straightforward. Each thread just gets frozen index of current first chunk which points right after last deleted element and returns actual key-value pair. If all keys was already deleted from current first chunk, then CBPQ needs restructure in order to find other "DELETION" chunk which becomes current first chunk. Once queue is restructured thread attempts to remove high-priority element again.\par
Once deletion is done thread increases value of frozen index using atomic F\&I instruction and as in all previous cases we can safely return.

\paragraph{Worst-case bounds estimation}\mbox{}\par 
According to information written above it becomes pretty obvious that the most expencive operation of our queue is restructure process.\par
Considering this fact, we can state that worst case of insertion in CBPQ is a situation when all elements inserted only into the first chunk as insertion into the first chunk requires queue to be restructured every time.\par
In deletion operation worst case appears when all elements have been deleted from the current first chunk and we need to execute the restructure of CBPQ in order to get key from other "DELETION" chunk.\newline\par
This section gives only a bird's eye overview of how CPBQ works. For any specific details we, again, refer you to the paper [1].

\subsection{Project Structure}\mbox{}\par
Project contains several directories:\textit{ benchmark/} directory which stores script to run benchmark and collect data for plots construction, \textit{plots/} and \textit{report/} directories for plots and report storage respectively and \textit{src/} directory which contains all benchmark's and CPBQ's sources.\par
Apart from that there are two files: \textit{Makefile} which serves to build and run benchmark and \textit{README.md} which contains project description.\par
\textbf{Read README.md carefully before use any Makefile targets!}

\section{Performance Evaluation}\mbox{}\par
The provided benchmark is a set of five tests each of which runs main CBPQ operations. Before each test we fill our queue with 10.000 elements in order to don't run our benchmark for empty queue.\par
Also we would like to notice that we, of course, tested our queue with much more elements, but due to time restriction we decided to stop on 10000 elements in order to our benchmark doesn't run longer than 1 minute.

\paragraph{Mixed workload with 50\%/50\% Insert()/DeteleMin() operation persantage}\mbox{}\par
During this test we add and remove 5000 key-value pairs into/from CBPQ.\par
The results are shown on \cref{fig:plot1}
\begin{center}
\textit{Not finished!}
\end{center}

\paragraph{Mixed workload with 80\% of DeteleMin() operation}\mbox{}\par
During this test we add 2000 and remove 8000 key-value pairs into/from CBPQ.\par
The results are shown on \cref{fig:plot2}.
\begin{center}
\textit{Not finished!}
\end{center}

\paragraph{Mixed workload with 20\% of DeteleMin() operation}\mbox{}\par
During this test we add 8000 and remove 2000 key-value pairs into/from CBPQ.\par
The results are shown on \cref{fig:plot3}.
\begin{center}
\textit{Not finished!}
\end{center}

\paragraph{Deletion-only workload}\mbox{}\par
During this test we remove 10000 key-value pairs from CBPQ.\par
The results are shown on \cref{fig:plot4}.
\begin{center}
\textit{Not finished!}
\end{center}

\paragraph{Insertion-only workload}\mbox{}\par
During this test we add 10000 key-value pairs into CBPQ.\par
The results are shown on \cref{fig:plot5}.
\begin{center}
\textit{Not finished!}
\end{center}

\begin{figure}[H]
  \centering
  \includegraphics{../plots/plot1.pdf}
  \caption{Mixed workload with 50\%/50\% \textit{Insert()}/\textit{DeteleMin()} operation persantage.}
  \label{fig:plot1}
\end{figure}

\begin{figure}[H]
  \centering
  \includegraphics{../plots/plot2.pdf}
  \caption{Mixed workload with 80\% of DeteleMin() operation.}
  \label{fig:plot2}
\end{figure}

\begin{figure}[H]
  \centering
  \includegraphics{../plots/plot3.pdf}
  \caption{Mixed workload with 20\% of DeteleMin() operation.}
  \label{fig:plot3}
\end{figure}

\begin{figure}[H]
  \centering
  \includegraphics{../plots/plot4.pdf}
  \caption{Deletion-only workload.}
  \label{fig:plot4}
\end{figure}

\begin{figure}[H]
  \centering
  \includegraphics{../plots/plot5.pdf}
  \caption{Insertion-only workload.}
  \label{fig:plot5}
\end{figure}

\section{Conclusions}\mbox{}\par
We presented a novel concurrent, linearizable, and lock-free design of the priority queue, called CBPQ, based on [1]. CBPQ cleverly combines the chunked linked-list and the performance advantage of the F\&I atomic instruction.
\begin{center}
\textit{Not finished!}
\end{center}

\section*{References}
1. Anastasia Braginsky, Nachshon Cohen, Erez Petrank: CBPQ: High Performance Lock-Free Priority Queue. Euro-Par 2016: 460-474

\end{document}
